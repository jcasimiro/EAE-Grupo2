% Options for packages loaded elsewhere
\PassOptionsToPackage{unicode}{hyperref}
\PassOptionsToPackage{hyphens}{url}
%
\documentclass[
]{article}
\usepackage{amsmath,amssymb}
\usepackage{iftex}
\ifPDFTeX
  \usepackage[T1]{fontenc}
  \usepackage[utf8]{inputenc}
  \usepackage{textcomp} % provide euro and other symbols
\else % if luatex or xetex
  \usepackage{unicode-math} % this also loads fontspec
  \defaultfontfeatures{Scale=MatchLowercase}
  \defaultfontfeatures[\rmfamily]{Ligatures=TeX,Scale=1}
\fi
\usepackage{lmodern}
\ifPDFTeX\else
  % xetex/luatex font selection
\fi
% Use upquote if available, for straight quotes in verbatim environments
\IfFileExists{upquote.sty}{\usepackage{upquote}}{}
\IfFileExists{microtype.sty}{% use microtype if available
  \usepackage[]{microtype}
  \UseMicrotypeSet[protrusion]{basicmath} % disable protrusion for tt fonts
}{}
\makeatletter
\@ifundefined{KOMAClassName}{% if non-KOMA class
  \IfFileExists{parskip.sty}{%
    \usepackage{parskip}
  }{% else
    \setlength{\parindent}{0pt}
    \setlength{\parskip}{6pt plus 2pt minus 1pt}}
}{% if KOMA class
  \KOMAoptions{parskip=half}}
\makeatother
\usepackage{xcolor}
\usepackage[margin=1in]{geometry}
\usepackage{color}
\usepackage{fancyvrb}
\newcommand{\VerbBar}{|}
\newcommand{\VERB}{\Verb[commandchars=\\\{\}]}
\DefineVerbatimEnvironment{Highlighting}{Verbatim}{commandchars=\\\{\}}
% Add ',fontsize=\small' for more characters per line
\usepackage{framed}
\definecolor{shadecolor}{RGB}{248,248,248}
\newenvironment{Shaded}{\begin{snugshade}}{\end{snugshade}}
\newcommand{\AlertTok}[1]{\textcolor[rgb]{0.94,0.16,0.16}{#1}}
\newcommand{\AnnotationTok}[1]{\textcolor[rgb]{0.56,0.35,0.01}{\textbf{\textit{#1}}}}
\newcommand{\AttributeTok}[1]{\textcolor[rgb]{0.13,0.29,0.53}{#1}}
\newcommand{\BaseNTok}[1]{\textcolor[rgb]{0.00,0.00,0.81}{#1}}
\newcommand{\BuiltInTok}[1]{#1}
\newcommand{\CharTok}[1]{\textcolor[rgb]{0.31,0.60,0.02}{#1}}
\newcommand{\CommentTok}[1]{\textcolor[rgb]{0.56,0.35,0.01}{\textit{#1}}}
\newcommand{\CommentVarTok}[1]{\textcolor[rgb]{0.56,0.35,0.01}{\textbf{\textit{#1}}}}
\newcommand{\ConstantTok}[1]{\textcolor[rgb]{0.56,0.35,0.01}{#1}}
\newcommand{\ControlFlowTok}[1]{\textcolor[rgb]{0.13,0.29,0.53}{\textbf{#1}}}
\newcommand{\DataTypeTok}[1]{\textcolor[rgb]{0.13,0.29,0.53}{#1}}
\newcommand{\DecValTok}[1]{\textcolor[rgb]{0.00,0.00,0.81}{#1}}
\newcommand{\DocumentationTok}[1]{\textcolor[rgb]{0.56,0.35,0.01}{\textbf{\textit{#1}}}}
\newcommand{\ErrorTok}[1]{\textcolor[rgb]{0.64,0.00,0.00}{\textbf{#1}}}
\newcommand{\ExtensionTok}[1]{#1}
\newcommand{\FloatTok}[1]{\textcolor[rgb]{0.00,0.00,0.81}{#1}}
\newcommand{\FunctionTok}[1]{\textcolor[rgb]{0.13,0.29,0.53}{\textbf{#1}}}
\newcommand{\ImportTok}[1]{#1}
\newcommand{\InformationTok}[1]{\textcolor[rgb]{0.56,0.35,0.01}{\textbf{\textit{#1}}}}
\newcommand{\KeywordTok}[1]{\textcolor[rgb]{0.13,0.29,0.53}{\textbf{#1}}}
\newcommand{\NormalTok}[1]{#1}
\newcommand{\OperatorTok}[1]{\textcolor[rgb]{0.81,0.36,0.00}{\textbf{#1}}}
\newcommand{\OtherTok}[1]{\textcolor[rgb]{0.56,0.35,0.01}{#1}}
\newcommand{\PreprocessorTok}[1]{\textcolor[rgb]{0.56,0.35,0.01}{\textit{#1}}}
\newcommand{\RegionMarkerTok}[1]{#1}
\newcommand{\SpecialCharTok}[1]{\textcolor[rgb]{0.81,0.36,0.00}{\textbf{#1}}}
\newcommand{\SpecialStringTok}[1]{\textcolor[rgb]{0.31,0.60,0.02}{#1}}
\newcommand{\StringTok}[1]{\textcolor[rgb]{0.31,0.60,0.02}{#1}}
\newcommand{\VariableTok}[1]{\textcolor[rgb]{0.00,0.00,0.00}{#1}}
\newcommand{\VerbatimStringTok}[1]{\textcolor[rgb]{0.31,0.60,0.02}{#1}}
\newcommand{\WarningTok}[1]{\textcolor[rgb]{0.56,0.35,0.01}{\textbf{\textit{#1}}}}
\usepackage{graphicx}
\makeatletter
\def\maxwidth{\ifdim\Gin@nat@width>\linewidth\linewidth\else\Gin@nat@width\fi}
\def\maxheight{\ifdim\Gin@nat@height>\textheight\textheight\else\Gin@nat@height\fi}
\makeatother
% Scale images if necessary, so that they will not overflow the page
% margins by default, and it is still possible to overwrite the defaults
% using explicit options in \includegraphics[width, height, ...]{}
\setkeys{Gin}{width=\maxwidth,height=\maxheight,keepaspectratio}
% Set default figure placement to htbp
\makeatletter
\def\fps@figure{htbp}
\makeatother
\setlength{\emergencystretch}{3em} % prevent overfull lines
\providecommand{\tightlist}{%
  \setlength{\itemsep}{0pt}\setlength{\parskip}{0pt}}
\setcounter{secnumdepth}{-\maxdimen} % remove section numbering
\ifLuaTeX
  \usepackage{selnolig}  % disable illegal ligatures
\fi
\IfFileExists{bookmark.sty}{\usepackage{bookmark}}{\usepackage{hyperref}}
\IfFileExists{xurl.sty}{\usepackage{xurl}}{} % add URL line breaks if available
\urlstyle{same}
\hypersetup{
  pdftitle={Projeto},
  hidelinks,
  pdfcreator={LaTeX via pandoc}}

\title{Projeto}
\author{}
\date{\vspace{-2.5em}2024-04-14}

\begin{document}
\maketitle

\begin{Shaded}
\begin{Highlighting}[]
\ControlFlowTok{if}\NormalTok{ (}\SpecialCharTok{!}\FunctionTok{requireNamespace}\NormalTok{(}\StringTok{"plot.matrix"}\NormalTok{, }\AttributeTok{quietly =} \ConstantTok{TRUE}\NormalTok{)) }\FunctionTok{install.packages}\NormalTok{(}\StringTok{"plot.matrix"}\NormalTok{)}
\ControlFlowTok{if}\NormalTok{ (}\SpecialCharTok{!}\FunctionTok{requireNamespace}\NormalTok{(}\StringTok{"dbscan"}\NormalTok{, }\AttributeTok{quietly =} \ConstantTok{TRUE}\NormalTok{)) }\FunctionTok{install.packages}\NormalTok{(}\StringTok{"dbscan"}\NormalTok{)}
\ControlFlowTok{if}\NormalTok{ (}\SpecialCharTok{!}\FunctionTok{requireNamespace}\NormalTok{(}\StringTok{"ggplot2"}\NormalTok{, }\AttributeTok{quietly =} \ConstantTok{TRUE}\NormalTok{)) }\FunctionTok{install.packages}\NormalTok{(}\StringTok{"ggplot2"}\NormalTok{)}
\ControlFlowTok{if}\NormalTok{ (}\SpecialCharTok{!}\FunctionTok{requireNamespace}\NormalTok{(}\StringTok{"rpart"}\NormalTok{, }\AttributeTok{quietly =} \ConstantTok{TRUE}\NormalTok{)) }\FunctionTok{install.packages}\NormalTok{(}\StringTok{"rpart"}\NormalTok{)}
\ControlFlowTok{if}\NormalTok{ (}\SpecialCharTok{!}\FunctionTok{requireNamespace}\NormalTok{(}\StringTok{"rpart.plot"}\NormalTok{, }\AttributeTok{quietly =} \ConstantTok{TRUE}\NormalTok{)) }\FunctionTok{install.packages}\NormalTok{(}\StringTok{"rpart.plot"}\NormalTok{)}
\ControlFlowTok{if}\NormalTok{ (}\SpecialCharTok{!}\FunctionTok{requireNamespace}\NormalTok{(}\StringTok{"caret"}\NormalTok{, }\AttributeTok{quietly =} \ConstantTok{TRUE}\NormalTok{)) }\FunctionTok{install.packages}\NormalTok{(}\StringTok{"caret"}\NormalTok{)}

\FunctionTok{library}\NormalTok{(plot.matrix)}
\FunctionTok{library}\NormalTok{(dbscan)}
\FunctionTok{library}\NormalTok{(ggplot2)}
\FunctionTok{library}\NormalTok{(rpart)}
\FunctionTok{library}\NormalTok{(rpart.plot)}
\FunctionTok{library}\NormalTok{(caret)}
\FunctionTok{library}\NormalTok{(dplyr)}


\NormalTok{plotfigure }\OtherTok{\textless{}\textless{}{-}} \ControlFlowTok{function}\NormalTok{(row,dataset)}
\NormalTok{\{}
\NormalTok{  X }\OtherTok{=} \ConstantTok{NULL}
  \ControlFlowTok{if}\NormalTok{(}\SpecialCharTok{!}\FunctionTok{is.null}\NormalTok{(}\FunctionTok{nrow}\NormalTok{(dataset)))}
\NormalTok{  \{}
\NormalTok{    X }\OtherTok{=} \FunctionTok{data.frame}\NormalTok{(}\FunctionTok{matrix}\NormalTok{(dataset[row,}\DecValTok{2}\SpecialCharTok{:}\DecValTok{785}\NormalTok{],}\AttributeTok{nrow=}\DecValTok{28}\NormalTok{))}
\NormalTok{  \}}
  \ControlFlowTok{else}
\NormalTok{  \{}
\NormalTok{    X }\OtherTok{=} \FunctionTok{data.frame}\NormalTok{(}\FunctionTok{matrix}\NormalTok{(dataset[row,}\DecValTok{2}\SpecialCharTok{:}\DecValTok{785}\NormalTok{],}\AttributeTok{nrow=}\DecValTok{28}\NormalTok{))}
\NormalTok{  \}}
\NormalTok{  m1 }\OtherTok{=} \FunctionTok{data.matrix}\NormalTok{(X)}
  \FunctionTok{plot}\NormalTok{(m1, }\AttributeTok{cex=}\FloatTok{0.5}\NormalTok{)}
\NormalTok{\}}
\end{Highlighting}
\end{Shaded}

\hypertarget{introduuxe7uxe3o}{%
\section{Introdução}\label{introduuxe7uxe3o}}

O presente trabalho tem como objetivo o desenvolvimento de um modelo que

A base de dados para o desenvolvimento do trabalho contém imagens de
caracteres alfanuméricos manuscritos, incluindo letras maiúsculas e
minúsculas, além de dígitos de 0 a 9. Cada imagem tem uma resolução de
28 × 28 = 784 pixeis. O trabalho será realizado para as letras D e F.

\begin{Shaded}
\begin{Highlighting}[]
\NormalTok{train\_data }\OtherTok{\textless{}\textless{}{-}} \FunctionTok{read.csv}\NormalTok{(}\StringTok{"emnist{-}balanced{-}train.csv"}\NormalTok{,}\AttributeTok{sep=}\StringTok{","}\NormalTok{,}\AttributeTok{header =} \ConstantTok{FALSE}\NormalTok{)}
\NormalTok{test\_data }\OtherTok{\textless{}\textless{}{-}} \FunctionTok{read.csv}\NormalTok{(}\StringTok{"emnist{-}balanced{-}test.csv"}\NormalTok{,}\AttributeTok{sep=}\StringTok{","}\NormalTok{,}\AttributeTok{header =} \ConstantTok{FALSE}\NormalTok{)}
\end{Highlighting}
\end{Shaded}

Nas figuras seguintes pode observar-se algumas das possíveis
representações da letra D e da letra F.

\begin{Shaded}
\begin{Highlighting}[]
\FunctionTok{par}\NormalTok{(}\AttributeTok{mfrow =} \FunctionTok{c}\NormalTok{(}\DecValTok{2}\NormalTok{, }\DecValTok{2}\NormalTok{))}

\FunctionTok{plotfigure}\NormalTok{(}\DecValTok{14}\NormalTok{,train\_data)}
\FunctionTok{plotfigure}\NormalTok{(}\DecValTok{60}\NormalTok{,train\_data)}
\FunctionTok{plotfigure}\NormalTok{(}\DecValTok{13}\NormalTok{,train\_data)}
\FunctionTok{plotfigure}\NormalTok{(}\DecValTok{4}\NormalTok{,train\_data)}
\end{Highlighting}
\end{Shaded}

\includegraphics{Projeto-Grupo_files/figure-latex/unnamed-chunk-3-1.pdf}

Dado que se pretende apenas trabalhar com as letras D e F, reduziu-se a
base de dados inicial para que esta apenas contenha as letras D e F.

\begin{Shaded}
\begin{Highlighting}[]
\NormalTok{label\_D }\OtherTok{\textless{}{-}} \DecValTok{13}
\NormalTok{label\_F }\OtherTok{\textless{}{-}} \DecValTok{15}

\NormalTok{filtered\_train\_data }\OtherTok{\textless{}{-}}\NormalTok{ train\_data[train\_data}\SpecialCharTok{$}\NormalTok{V1 }\SpecialCharTok{\%in\%} \FunctionTok{c}\NormalTok{(label\_D, label\_F), ]}
\NormalTok{filtered\_test\_data }\OtherTok{\textless{}{-}}\NormalTok{ test\_data[test\_data}\SpecialCharTok{$}\NormalTok{V1 }\SpecialCharTok{\%in\%} \FunctionTok{c}\NormalTok{(label\_D, label\_F), ]}

\FunctionTok{head}\NormalTok{(filtered\_train\_data[, }\DecValTok{1}\SpecialCharTok{:}\DecValTok{5}\NormalTok{], }\DecValTok{5}\NormalTok{)}
\end{Highlighting}
\end{Shaded}

\begin{verbatim}
##    V1 V2 V3 V4 V5
## 4  15  0  0  0  0
## 13 15  0  0  0  0
## 14 13  0  0  0  0
## 53 15  0  0  0  0
## 73 13  0  0  0  0
\end{verbatim}

\hypertarget{anuxe1lise-exploratuxf3ria}{%
\section{Análise Exploratória}\label{anuxe1lise-exploratuxf3ria}}

Com o objetivo de distinguir as duas letras, é necessário determinar uma
zona de relevância, ou seja, o conjunto de pixeis que determinam com
elevada probabilidade se na imagem está representada a letra D ou a
letra F.

Com este intuito foi construída uma árvore de decisão, representada na
figura seguinte.

\begin{Shaded}
\begin{Highlighting}[]
\FunctionTok{set.seed}\NormalTok{(}\DecValTok{123}\NormalTok{)}

\NormalTok{modelo\_arvore }\OtherTok{\textless{}{-}} \FunctionTok{rpart}\NormalTok{(V1 }\SpecialCharTok{\textasciitilde{}}\NormalTok{ ., }\AttributeTok{data =}\NormalTok{ filtered\_train\_data, }\AttributeTok{method =} \StringTok{"class"}\NormalTok{)}

\FunctionTok{rpart.plot}\NormalTok{(modelo\_arvore, }\AttributeTok{main=}\StringTok{"Árvore de Decisão {-} \textquotesingle{}D\textquotesingle{} vs \textquotesingle{}F\textquotesingle{}"}\NormalTok{,}
           \AttributeTok{box.palette =} \StringTok{"RdBu"}\NormalTok{)}
\end{Highlighting}
\end{Shaded}

\includegraphics{Projeto-Grupo_files/figure-latex/unnamed-chunk-5-1.pdf}

A árvore de decisão indica como pixel mais influente na diferenciação
das letras o pixel 554. Se o valor do pixel for superior ou igual a 1,
classifica-se a imagem como D. No entanto, se o pixel tiver um valor
inferior a 1, o pixel 640 passa a ter influência para classificar a
imagem. Isto significa que o pixel 554 é suficiente para classificar a
imagem como D, mas não para classificar a imagem como F. A classificação
da imagem como a letra F passa por analisar os pixeis 554, 640 e 496.

Seguidamente, estão representados os gráficos das densidades para os
valores do pixel 554, para a letra D e para a letra F. A representação
da letra D no gráfico é feita pela cor azul, enquanto que a letra F é
exibida pela cor vermelha. A linha vertical a tracejado indica quando o
valor do pixel é igual a 1. É visível que para imagens da letra F,
existe uma predisposição maior para que o valores do pixel 554 sejam
menores, enquanto que, para imagens da letra D, o pixel assume vários
valores. (\emph{comentar com árvore})

\includegraphics{Projeto-Grupo_files/figure-latex/unnamed-chunk-6-1.pdf}

Pela caixa-de-bigodes seguinte, é perceptível que para valores do pixel
554 inferiores a 1, existe uma distinção entre a letra D e F. Podemos
observar que para as imagens da letra F, é mais comum o pixel 554
apresentar valores inferiores a 1 e para as imagens da letra D, é mais
comum o pixel 554 apresentar valores superiores a 1.

\begin{Shaded}
\begin{Highlighting}[]
\FunctionTok{boxplot}\NormalTok{(}\FunctionTok{as.matrix}\NormalTok{(filtered\_train\_data[,}\DecValTok{554}\SpecialCharTok{:}\DecValTok{554}\NormalTok{]) }\SpecialCharTok{\textasciitilde{}}\NormalTok{ character\_labels,}
        \AttributeTok{col=}\FunctionTok{c}\NormalTok{(}\StringTok{"\#FF9966"}\NormalTok{, }\StringTok{"\#996633"}\NormalTok{), }
        \AttributeTok{las=}\DecValTok{2}\NormalTok{,}
        \AttributeTok{xlab=}\StringTok{"Letras"}\NormalTok{,}
        \AttributeTok{ylab=}\StringTok{"Intensidade dos pixeis"}\NormalTok{,}
        \CommentTok{\#outline=FALSE}
\NormalTok{)}
\FunctionTok{title}\NormalTok{(}\AttributeTok{main =} \StringTok{"Distribuição dos Valores dos Pixels para \textquotesingle{}D\textquotesingle{} e \textquotesingle{}F\textquotesingle{}"}\NormalTok{, }\AttributeTok{cex.main =} \FloatTok{0.7}\NormalTok{)}
\end{Highlighting}
\end{Shaded}

\includegraphics{Projeto-Grupo_files/figure-latex/unnamed-chunk-7-1.pdf}

\begin{Shaded}
\begin{Highlighting}[]
\FunctionTok{par}\NormalTok{(}\AttributeTok{mfrow =} \FunctionTok{c}\NormalTok{(}\DecValTok{1}\NormalTok{, }\DecValTok{2}\NormalTok{))}

\FunctionTok{ggplot}\NormalTok{(combined\_data, }\FunctionTok{aes}\NormalTok{(}\AttributeTok{x =}\NormalTok{ V640, }\AttributeTok{fill =}\NormalTok{ group)) }\SpecialCharTok{+}
  \FunctionTok{geom\_density}\NormalTok{(}\AttributeTok{alpha =} \FloatTok{0.6}\NormalTok{) }\SpecialCharTok{+}
  \FunctionTok{scale\_fill\_manual}\NormalTok{(}\AttributeTok{values =} \FunctionTok{c}\NormalTok{(}\StringTok{"blue"}\NormalTok{, }\StringTok{"red"}\NormalTok{)) }\SpecialCharTok{+}
  \FunctionTok{theme\_minimal}\NormalTok{() }\SpecialCharTok{+}
  \FunctionTok{labs}\NormalTok{(}\AttributeTok{x =} \StringTok{"Valor do Pixel"}\NormalTok{, }\AttributeTok{y =} \StringTok{"Densidade"}\NormalTok{, }\AttributeTok{title =} \StringTok{"Gráfico de Dendidade para o Pixel 554"}\NormalTok{) }\SpecialCharTok{+}
  \FunctionTok{guides}\NormalTok{(}\AttributeTok{fill =} \FunctionTok{guide\_legend}\NormalTok{(}\AttributeTok{title =} \StringTok{"Grupo"}\NormalTok{)) }\SpecialCharTok{+} 
  \FunctionTok{geom\_vline}\NormalTok{(}\AttributeTok{xintercept =} \DecValTok{1}\NormalTok{, }\AttributeTok{linetype =} \StringTok{"dashed"}\NormalTok{, }\AttributeTok{color =} \StringTok{"black"}\NormalTok{) }\SpecialCharTok{+}
  \FunctionTok{coord\_cartesian}\NormalTok{(}\AttributeTok{ylim =} \FunctionTok{c}\NormalTok{(}\DecValTok{0}\NormalTok{, }\FloatTok{0.25}\NormalTok{))}
\end{Highlighting}
\end{Shaded}

\includegraphics{Projeto-Grupo_files/figure-latex/unnamed-chunk-8-1.pdf}

\begin{Shaded}
\begin{Highlighting}[]
\CommentTok{\#{-}{-}{-}{-}{-}{-}{-}{-}{-}{-}{-}{-}{-}{-}{-}{-}{-}}

\FunctionTok{ggplot}\NormalTok{(combined\_data, }\FunctionTok{aes}\NormalTok{(}\AttributeTok{x =}\NormalTok{ V496, }\AttributeTok{fill =}\NormalTok{ group)) }\SpecialCharTok{+}
  \FunctionTok{geom\_density}\NormalTok{(}\AttributeTok{alpha =} \FloatTok{0.6}\NormalTok{) }\SpecialCharTok{+}
  \FunctionTok{scale\_fill\_manual}\NormalTok{(}\AttributeTok{values =} \FunctionTok{c}\NormalTok{(}\StringTok{"blue"}\NormalTok{, }\StringTok{"red"}\NormalTok{)) }\SpecialCharTok{+}
  \FunctionTok{theme\_minimal}\NormalTok{() }\SpecialCharTok{+}
  \FunctionTok{labs}\NormalTok{(}\AttributeTok{x =} \StringTok{"Valor do Pixel"}\NormalTok{, }\AttributeTok{y =} \StringTok{"Densidade"}\NormalTok{, }\AttributeTok{title =} \StringTok{"Gráfico de Dendidade para o Pixel 554"}\NormalTok{) }\SpecialCharTok{+}
  \FunctionTok{guides}\NormalTok{(}\AttributeTok{fill =} \FunctionTok{guide\_legend}\NormalTok{(}\AttributeTok{title =} \StringTok{"Grupo"}\NormalTok{)) }\SpecialCharTok{+} 
  \FunctionTok{geom\_vline}\NormalTok{(}\AttributeTok{xintercept =} \DecValTok{1}\NormalTok{, }\AttributeTok{linetype =} \StringTok{"dashed"}\NormalTok{, }\AttributeTok{color =} \StringTok{"black"}\NormalTok{) }\SpecialCharTok{+}
  \FunctionTok{coord\_cartesian}\NormalTok{(}\AttributeTok{ylim =} \FunctionTok{c}\NormalTok{(}\DecValTok{0}\NormalTok{, }\FloatTok{0.1}\NormalTok{))}
\end{Highlighting}
\end{Shaded}

\includegraphics{Projeto-Grupo_files/figure-latex/unnamed-chunk-8-2.pdf}

\begin{Shaded}
\begin{Highlighting}[]
\NormalTok{sums\_D }\OtherTok{\textless{}{-}} \FunctionTok{colSums}\NormalTok{(filtered\_train\_data[filtered\_train\_data}\SpecialCharTok{$}\NormalTok{V1 }\SpecialCharTok{==}\NormalTok{ label\_D, }\SpecialCharTok{{-}}\DecValTok{1}\NormalTok{])}
\NormalTok{sums\_F }\OtherTok{\textless{}{-}} \FunctionTok{colSums}\NormalTok{(filtered\_train\_data[filtered\_train\_data}\SpecialCharTok{$}\NormalTok{V1 }\SpecialCharTok{==}\NormalTok{ label\_F, }\SpecialCharTok{{-}}\DecValTok{1}\NormalTok{])}

\NormalTok{differences }\OtherTok{\textless{}{-}} \FunctionTok{abs}\NormalTok{(sums\_D }\SpecialCharTok{{-}}\NormalTok{ sums\_F)}


\NormalTok{matrix\_dim }\OtherTok{\textless{}{-}} \DecValTok{28}
\NormalTok{differences\_matrix }\OtherTok{\textless{}{-}} \FunctionTok{matrix}\NormalTok{(differences, }\AttributeTok{nrow =}\NormalTok{ matrix\_dim, }\AttributeTok{ncol =}\NormalTok{ matrix\_dim)}


\FunctionTok{image}\NormalTok{(}\DecValTok{1}\SpecialCharTok{:}\DecValTok{28}\NormalTok{, }\DecValTok{1}\SpecialCharTok{:}\DecValTok{28}\NormalTok{, differences\_matrix,}
      \AttributeTok{col =} \FunctionTok{cm.colors}\NormalTok{(}\DecValTok{20}\NormalTok{),}
      \AttributeTok{xlab =} \StringTok{"Pixeis"}\NormalTok{, }\AttributeTok{ylab =} \StringTok{"Pixeis"}\NormalTok{,}
      \AttributeTok{main =} \StringTok{"Diferença dos Valores dos Pixeis"}\NormalTok{)}
\end{Highlighting}
\end{Shaded}

\includegraphics{Projeto-Grupo_files/figure-latex/unnamed-chunk-9-1.pdf}

\end{document}
